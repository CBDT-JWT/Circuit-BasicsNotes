\documentclass[11pt]{beamer}

% 使用 ctex 宏包支持中文
\usepackage[UTF8]{ctex}
\usepackage{circuitikz}
% 设置主题颜色为清华紫
\definecolor{TsinghuaPurple}{RGB}{79, 0, 128}
\setbeamercolor{structure}{fg=TsinghuaPurple}

% 设置页面背景为白色
\setbeamercolor{background canvas}{bg=white}

% 主题设置
\usecolortheme{default}
\useinnertheme{rounded}
\useoutertheme{infolines}

% 其他常用宏包
\usepackage{graphicx} % 插入图片
\usepackage{booktabs} % 表格
\usepackage{hyperref} % 超链接

% 修改目录样式,去掉圆圈
\setbeamertemplate{section in toc}{\inserttocsection}
\setbeamertemplate{subsection in toc}{\inserttocsubsection}

% 全局设置 circuitikz 电阻样式为 European
\ctikzset{resistor = european}
% 全局设置 circuitikz 电感样式为 American
\ctikzset{inductor = american}

% 设置数学字体为默认的 Computer Modern
\usefonttheme[onlymath]{serif}

% 标题信息
\title{电子电路与系统基础(II)参考讲义}
\author[江玮陶 \,电子系学生科协 学培部]{江玮陶\quad\textit{电子系学生科协学培部}}
\date{\today}

\begin{document}

% 标题页
\begin{frame}
    \titlepage
\end{frame}

% 目录页
\begin{frame}{目录}
    \tableofcontents
\end{frame}

% 示例内容
\section{绪论}
\begin{frame}{如何学电电?}
    \begin{columns}
        \column{0.5\textwidth}
    这是我们中学学习的电路。
    \column{0.5\textwidth}
    \begin{figure}[!ht]
\centering
\resizebox{.8\textwidth}{!}{%
\begin{circuitikz}
\tikzstyle{every node}=[font=\large]
\draw (6.25,12.75) to[rmeter, t=V] (8.75,12.75);
\draw (3.75,11.5) to[battery1,l=$U$] (3.75,9);
\draw (3.75,11.5) to[rmeter, t=A] (6.25,11.5);
\draw (6.25,11.5) to[lamp] (8.75,11.5);
\draw (8.75,11.5) to[potentiometer,l={ \large $R_p$}] (8.75,9);
\draw (3.75,9) to[short] (8.75,9);
\draw (6.25,12.75) to[short] (6.25,11.5);
\draw (8.75,12.75) to[short] (8.75,11.5);
\draw (9.25,10.25) to[short] (9.5,10.25);
\draw (9.5,10.25) to[short] (9.5,11.25);
\draw (9.5,11.25) to[short] (8.75,11.25);
\draw (9.5,10.25) to[short] (9,10.25);
\end{circuitikz}
}%
\label{fig:my_label}
\end{figure}
\end{columns}
\end{frame}

\begin{frame}{如何学电电?}
    \begin{columns}
        \column{0.5\textwidth}
        这是我们现在学习的电路。
        \column{.5\textwidth}
        \begin{figure}
    \centering
    \includegraphics[width=.9\linewidth]{figures/ua741.png}
        \end{figure}
        \begin{figure}
            \centering
            \includegraphics[width=.9\linewidth]{figures/image.png}
            \end{figure}
    \end{columns}
\end{frame}
\begin{frame}{如何学电电?}
    \begin{center}
        电电的脉络是什么?
    \end{center}
    \begin{columns}
        \column{.32\textwidth}
            \begin{center}
                {\Large\textcolor{TsinghuaPurple}{电子}}
                \vspace{0.5cm}

                电阻,电容,电感

                BJT,MOS,二极管
                
                运放,互感,受控源

                ······
            \end{center}
        \column{.32\textwidth}
            \begin{center}
                {\Large\textcolor{TsinghuaPurple}{电路}}
                \vspace{0.5cm}
                
                单管多管放大器

                无源有源滤波器

                负电阻,振荡器

                ······
            \end{center}
        \column{.32\textwidth}
            \begin{center}
                {\Large\textcolor{TsinghuaPurple}{系统}}
                \vspace{0.5cm}
                
                电路等效方法

                基尔霍夫定律

                时频域分析方法

                负反馈和正反馈
                
                ······
            \end{center}

        
    \end{columns}
    \begin{center}
            \Large 基础:基于矩阵和器件方程的描述方法
    \end{center}
\end{frame}

\section{元件器件}

\begin{frame}{电容电感}

    属于基础内容,请务必记牢两个元件的时频域元件方程和表达式。

    \begin{minipage}[t]{.95\textwidth}
        \begin{columns}
            \column{.20\textwidth}
        \begin{figure}[t]
            \centering
\resizebox{.8\textwidth}{!}{%
\begin{circuitikz}
\tikzstyle{every node}=[font=\LARGE]
\draw (3.75,18.25) to[C,l={ \LARGE $C$}] (3.75,15.75);
\draw (3.75,18.25) to[short, -o] (5,18.25) ;
\draw (3.75,15.75) to[short, -o] (5,15.75) ;
\draw [->, >=Stealth] (5,17.75) -- (5,16.25);
\draw [->, >=Stealth] (4.75,18.25) -- (4.25,18.25);
\node [font=\LARGE] at (4.25,18.75) {$i_C(t)$};
\node [font=\LARGE] at (6,17) {$v_C(t)$};
\end{circuitikz}
}%
\end{figure}
\column{.80\textwidth}
    \begin{itemize}
        \item 时域:$i_C(t)=C\frac{d}{dt}v_C(t)$, $v_C(t)=\frac{1}{C} \int_{-\infty}^t i_C(\tau)d\tau$
        \item 频域:$I_C=j\omega C V_C$, $V_C=\frac{1}{j\omega C} I_C$
        \item 容\textcolor{red}{纳}:$B_C=\omega C$
    \end{itemize}
\end{columns}
\end{minipage}
\begin{minipage}[t]{.95\textwidth}
        \begin{columns}
            \column{.20\textwidth}
        \begin{figure}[t]
            \centering
\resizebox{.8\textwidth}{!}{%
\begin{circuitikz}
\tikzstyle{every node}=[font=\LARGE]
\draw (3.75,18.25) to[american inductor,l={ \LARGE $L$}] (3.75,15.75);
\draw (3.75,18.25) to[short, -o] (5,18.25) ;
\draw (3.75,15.75) to[short, -o] (5,15.75) ;
\draw [->, >=Stealth] (5,17.75) -- (5,16.25);
\draw [->, >=Stealth] (4.75,18.25) -- (4.25,18.25);
\node [font=\LARGE] at (4.25,18.75) {$i_L(t)$};
\node [font=\LARGE] at (6,17) {$v_L(t)$};
\end{circuitikz}
}%
\end{figure}
\column{.80\textwidth}
    \begin{itemize}
        \item 时域:$v_L(t)=L\frac{d}{dt}i_L(t)$, $i_L(t)=\frac{1}{L} \int_{-\infty}^t v_L(\tau)d\tau$
        \item 频域:$V_L=j\omega L I_L$, $I_L=\frac{1}{j\omega L} V_L$
        \item 感\textcolor{red}{抗}:$X_L=\omega L$
    \end{itemize}
\end{columns}
\end{minipage}
\emph{Laplace}变换:$s=\sigma+j\omega, \dfrac{\mathrm{d}}{\mathrm{d}t} \rightleftharpoons s, \int_{-\infty}^t(\cdot)\mathrm{d}t\rightleftharpoons \frac{1}{s}$;特别研究虚轴上的情形($\sigma=0,\textcolor{red}{s=j\omega}$)即为\emph{Fourier}变换("频域特性")。电电课不需要, 也\textbf{不很建议}掌握\emph{Laplace}变换的具体形式,但要知道上述时频对应关系。
\end{frame}

\begin{frame}{互感变压器}
    物理模型不太重要,但还是了解一下(尤其是$L$和$N$的关系)。
    \begin{columns}
        \column{.3\textwidth}
        \begin{figure}[!ht]
\centering
\resizebox{.9\textwidth}{!}{%
\begin{circuitikz}
\tikzstyle{every node}=[font=\LARGE]
\draw (5,18.25) to[L ] (5,15.75);
\draw (6.25,15.75) to[L ] (6.25,18.25);
\draw (5,18.25) to[short, -o] (3.75,18.25) ;
\draw (5,15.75) to[short, -o] (3.75,15.75) ;
\draw (6.25,18.25) to[short, -o] (7.5,18.25) ;
\draw (6.25,15.75) to[short, -o] (7.5,15.75) ;
\node [font=\LARGE] at (7,17) {$L_2$};
\node [font=\LARGE] at (4.25,17) {$L_1$};
\draw [<->, >=Stealth] (5,18.25)--(6.25,18.25)node[pos=0.5, fill=white]{$M$};
\node at (5.25,18) [circ] {};
\node at (6,18) [circ] {};
\end{circuitikz}
}%
\label{fig:my_label}
\end{figure}
$$
\begin{cases}
    L_1 = N_1^2\Xi\\
    L_2 = N_2^2\Xi\\
    M = k N_1 N_2 \Xi \\
\end{cases}
$$
$$
M= k\sqrt{L_1 L_2}
$$
\column{0.7\textwidth}
同名端:流入电流使得\textcolor{TsinghuaPurple}{\textbf{磁通加强}}的两个点(黑点)
\begin{itemize}
    \item 物理参数:$N_1,N_2$(匝数),$\Xi$(磁导),$k$(耦合系数)
    \item 电路参数:$L_1,L_2$(自感),$M$(互感)
\end{itemize}
$\Xi=\mu \frac{S}{p}$:磁导,$\mu$磁导率,$S$截面积,$p$磁路长度

$k \in [0,1]$:耦合系数,表示磁通量链接百分比

\begin{center}
    \textcolor{TsinghuaPurple}{关键:熟练运用等效电路(T型等效/励漏磁等效)和阻抗变换原理化简电路}
\end{center}
\end{columns}
\end{frame}
\begin{frame}{互感变压器}
    \begin{columns}
        \column{.3\textwidth}
        \begin{figure}[!ht]
\centering
\resizebox{.9\textwidth}{!}{%
\begin{circuitikz}
\tikzstyle{every node}=[font=\LARGE]
\draw (5,18.25) to[L ] (5,15.75);
\draw (6.25,15.75) to[L ] (6.25,18.25);
\draw (5,18.25) to[short, -o] (3.75,18.25) ;
\draw (5,15.75) to[short, -o] (3.75,15.75) ;
\draw (6.25,18.25) to[short, -o] (7.5,18.25) ;
\draw (6.25,15.75) to[short, -o] (7.5,15.75) ;
\draw [color={rgb,255:red,255; green,38; blue,0},thick](6.25,15.75) to (5,15.75);
\node [font=\LARGE] at (7,17) {$L_2$};
\node [font=\LARGE] at (4.25,17) {$L_1$};
\draw [<->, >=Stealth] (5,18.25)--(6.25,18.25)node[pos=0.5, fill=white]{$M$};
\node at (5.25,18) [circ] {};
\node at (6,18) [circ] {};
\end{circuitikz}
}%
\label{fig:my_label}
\end{figure}
$$
\begin{cases}
    L_1 = N_1^2\Xi\\
    L_2 = N_2^2\Xi\\
    M = k N_1 N_2 \Xi \\
\end{cases}
$$
$$
M= k\sqrt{L_1 L_2}
$$
\column{0.7\textwidth}
$$
\begin{bmatrix}
    v_1(t)\\
    v_2(t)
\end{bmatrix}=\begin{bmatrix}
    L_1 & M \\
    M & L_2
\end{bmatrix}\frac{\mathrm{d}}{\mathrm{d}t}\begin{bmatrix}
    i_1(t)\\
    i_2(t)
\end{bmatrix}\Rightarrow\text{时域方程}
$$
$$
\begin{bmatrix}
    v_1(s)\\
    v_2(s)
\end{bmatrix}=\begin{bmatrix}
    sL_1 & sM \\
    sM & sL_2
\end{bmatrix}\begin{bmatrix}
    i_1(s)\\
    i_2(s)
\end{bmatrix}\Rightarrow\text{频域方程}
$$

\begin{figure}[!ht]
\centering
\resizebox{.4\textwidth}{!}{%
\begin{circuitikz}
\tikzstyle{every node}=[font=\LARGE]
\draw (2.5,17) to[L,l={ \LARGE $L_1-M$} ] (5,17);
\draw (5,17) to[L,l={ \LARGE $M$} ] (5,14.5);
\draw (5,17) to[L,l={ \LARGE $L_2-M$} ] (7.5,17);
\draw (2.5,17) to[short, -o] (2.25,17) ;
\draw (5,14.5) to[short, -o] (2.25,14.5) ;
\draw (5,14.5) to[short, -o] (7.75,14.5) ;
\draw (7.5,17) to[short, -o] (7.75,17) ;
\end{circuitikz}
}%
\end{figure}
\begin{center}
两边共地:T型等效
\end{center}
\end{columns}
\end{frame}
\begin{frame}{互感变压器}
    \begin{columns}
        \column{.3\textwidth}
        \begin{figure}[!ht]
\centering
\resizebox{.9\textwidth}{!}{%
\begin{circuitikz}
\tikzstyle{every node}=[font=\LARGE]
\draw (5,18.25) to[L ] (5,15.75);
\draw (6.25,15.75) to[L ] (6.25,18.25);
\draw (5,18.25) to[short, -o] (3.75,18.25) ;
\draw (5,15.75) to[short, -o] (3.75,15.75) ;
\draw (6.25,18.25) to[short, -o] (7.5,18.25) ;
\draw (6.25,15.75) to[short, -o] (7.5,15.75) ;
\node [font=\LARGE] at (7,17) {$L_2$};
\node [font=\LARGE] at (4.25,17) {$L_1$};
\draw [<->, >=Stealth] (5,18.25)--(6.25,18.25)node[pos=0.5, fill=white]{$M$};
\node at (5.25,18) [circ] {};
\node at (6,18) [circ] {};
\end{circuitikz}
}%
\label{fig:my_label}
\end{figure}
$$
\begin{cases}
    L_1 = N_1^2\Xi\\
    L_2 = N_2^2\Xi\\
    M = k N_1 N_2 \Xi \\
\end{cases}
$$
$$
M= k\sqrt{L_1 L_2}
$$
\column{0.7\textwidth}
$$
\begin{bmatrix}
    v_1(t)\\
    v_2(t)
\end{bmatrix}=\begin{bmatrix}
    L_1 & M \\
    M & L_2
\end{bmatrix}\frac{\mathrm{d}}{\mathrm{d}t}\begin{bmatrix}
    i_1(t)\\
    i_2(t)
\end{bmatrix}\Rightarrow\text{时域方程}
$$
$$
\begin{bmatrix}
    v_1(s)\\
    v_2(s)
\end{bmatrix}=\begin{bmatrix}
    sL_1 & sM \\
    sM & sL_2
\end{bmatrix}\begin{bmatrix}
    i_1(s)\\
    i_2(s)
\end{bmatrix}\Rightarrow\text{频域方程}
$$

\begin{figure}[!ht]
\centering
\resizebox{.6\textwidth}{!}{%
\begin{circuitikz}
\tikzstyle{every node}=[font=\LARGE]
\draw (2.5,17) to[L,l={ \LARGE $L_1-M$} ] (5,17);
\draw (5,17) to[L,l={ \LARGE $M$} ] (5,14.5);
\draw (5,17) to[L,l={ \LARGE $L_2-M$} ] (7.5,17);
\draw (2.5,17) to[short, -o] (2.25,17) ;
\draw (5,14.5) to[short, -o] (2.25,14.5) ;
% \draw (7.5,17) to[short, -o] (7.5,17) ;
% \draw (7.5,17.25) to[short, -o] (7.5,17.25) ;
\draw (5,14.5) to[short] (7.5,14.5);
\draw [ color={rgb,255:red,255; green,38; blue,0}, ](7.5,17) to[L ] (7.5,14.5);
\draw [ color={rgb,255:red,255; green,38; blue,0}, ](8.75,14.5) to[L ] (8.75,17);
\draw (8.75,14.5) to[short, -o] (9.75,14.5) ;
\draw (8.75,17) to[short, -o] (9.75,17) ;
\node [font=\LARGE, color={rgb,255:red,255; green,38; blue,0}] at (7,15.75) {$\infty$};
\node [font=\LARGE, color={rgb,255:red,255; green,38; blue,0}] at (9.25,15.75) {$\infty$};
\node [font=\LARGE, color={rgb,255:red,255; green,38; blue,0}] at (8.12,17.25) {$1:1$};
\end{circuitikz}
}%

\end{figure}
\begin{center}
两边不共地:T型等效加理想变压器
\end{center}
\end{columns}
\end{frame}
\begin{frame}{互感变压器}
    \begin{columns}
        \column{.5\textwidth}
        \begin{figure}[!ht]
\centering
\resizebox{.8\textwidth}{!}{%
\begin{circuitikz}
\tikzstyle{every node}=[font=\LARGE]
% \draw (2.5,17) to[L,l={ \LARGE $L_1-M$} ] (5,17);
% \draw (5,17) to[L,l={ \LARGE $M$} ] (5,14.5);
\draw (4.75,17) to[L,l={ \LARGE $L_1(1-k^2)$} ] (7.25,17);
\draw (4.75,17) to[short, -o] (4.25,17) ;
\draw (5,14.5) to[short, -o] (4.25,14.5) ;
\draw(7.25,17) to[short] (7.5,17) ;
% \draw (7.5,17) to[short, -o] (7.5,17) ;
% \draw (7.5,17.25) to[short, -o] (7.5,17.25) ;
\draw (5,14.5) to[short] (7.5,14.5);
\draw [ color={rgb,255:red,255; green,38; blue,0}, ](7.5,17) to[L ] (7.5,14.5);
\draw [ color={rgb,255:red,255; green,38; blue,0}, ](8.75,14.5) to[L ] (8.75,17);
\draw (8.75,14.5) to[short, -o] (11.25,14.5) ;
\draw (8.75,17) to[short, -o] (11.25,17) ;
\node [font=\LARGE, color={rgb,255:red,255; green,38; blue,0}] at (7,15.75) {$\infty$};
\node [font=\LARGE, color={rgb,255:red,255; green,38; blue,0}] at (9.25,15.75) {$\infty$};
\node [font=\LARGE, color={rgb,255:red,255; green,38; blue,0}] at (8.12,17.25) {$M:L_2$};
\draw (9.75,17) to[L,l={ \LARGE $L_2$} ] (9.75,14.5);
\end{circuitikz}
}%
\end{figure}

\begin{center}
漏磁励磁模型(h参量)
\end{center}
\column{.5\textwidth}
        \begin{figure}[!ht]
\centering
\resizebox{.8\textwidth}{!}{%
\begin{circuitikz}
\tikzstyle{every node}=[font=\LARGE]
% \draw (2.5,17) to[L,l={ \LARGE $L_1-M$} ] (5,17);
% \draw (5,17) to[L,l={ \LARGE $M$} ] (5,14.5);
\draw (6.5,14.5) to[L,l={ \LARGE $L_1$} ] (6.5,17);
\draw (7.5,17) to[short, -o] (4.75,17) ;
\draw (5,14.5) to[short, -o] (4.75,14.5) ;
% \draw(7.25,17) to[short] (7.5,17) ;
% \draw (7.5,17) to[short, -o] (7.5,17) ;
% \draw (7.5,17.25) to[short, -o] (7.5,17.25) ;
\draw (5,14.5) to[short] (7.5,14.5);
\draw [ color={rgb,255:red,255; green,38; blue,0}, ](7.5,17) to[L ] (7.5,14.5);
\draw [ color={rgb,255:red,255; green,38; blue,0}, ](8.75,14.5) to[L ] (8.75,17);
\draw (8.75,14.5) to[short, -o] (12,14.5) ;
% \draw (8.75,17) to[short, -o] (11.25,17) ;
\node [font=\LARGE, color={rgb,255:red,255; green,38; blue,0}] at (7,15.75) {$\infty$};
\node [font=\LARGE, color={rgb,255:red,255; green,38; blue,0}] at (9.25,15.75) {$\infty$};
\node [font=\LARGE, color={rgb,255:red,255; green,38; blue,0}] at (8.12,17.25) {$L_1:M$};
% \draw (9.75,17) to[L,l={ \LARGE $L_2$} ] (9.75,14.5);
\draw(8.75,17)--(9.5,17)to[L,l={ \LARGE $L_2(1-k^2)$} ](11.5,17)to[short,-o](12,17);
\end{circuitikz}
}%
\end{figure}

\begin{center}
励磁漏磁模型(g参量)
\end{center}
    \end{columns}
    \vspace{1cm}
$$
M=k\sqrt{L_1 L_2} \Rightarrow k^2=\frac{M^2}{L_1 L_2}
$$

\end{frame}

\begin{frame}{理想变压器}
    \begin{columns}
        \column{.25\textwidth}
        \begin{figure}[!ht]
\centering
\resizebox{1\textwidth}{!}{%
\begin{circuitikz}
\tikzstyle{every node}=[font=\LARGE]
\draw (7.5,17) to[short, -o] (5.75,17) ;
\draw (7.5,14.5) to[short, -o] (5.75,14.5) ;
\draw [ color={rgb,255:red,255; green,38; blue,0}, ](7.5,17) to[L ] (7.5,14.5);
\draw [ color={rgb,255:red,255; green,38; blue,0}, ](8.75,14.5) to[L ] (8.75,17);
\node [font=\LARGE, color={rgb,255:red,255; green,38; blue,0}] at (7,15.75) {$\infty$};
\node [font=\LARGE, color={rgb,255:red,255; green,38; blue,0}] at (9.25,15.75) {$\infty$};
\node [font=\LARGE, color={rgb,255:red,255; green,38; blue,0}] at (8.25,17.5) {$N_1:N_2$};
\draw (10,17) to[R,l={ \LARGE $Z_L$}] (10,14.5);
\draw (8.75,17) to[short] (10,17);
\draw (8.75,14.5) to[short] (10,14.5);
\draw [line width=1pt, dashed] (6.25,17.75) -- (6.25,13.5);
\draw [line width=1pt, ->, >=Stealth] (5.75,15) -- (6.75,15);
\draw [line width=1pt, short] (5.75,15) -- (5.75,13.25);
\end{circuitikz}
}%
\end{figure}
\begin{figure}[!ht]
\centering
\resizebox{1\textwidth}{!}{%
\begin{circuitikz}
\tikzstyle{every node}=[font=\LARGE]
\draw (8.75,17) to[short, -o] (5.75,17) ;
\draw (8.75,14.5) to[short, -o] (5.75,14.5) ;
\draw (10,17) to[R,l={ \LARGE $Z_L'$}] (10,14.5);
\draw (8.75,17) to[short] (10,17);
\draw (8.75,14.5) to[short] (10,14.5);
\draw [line width=1pt, dashed] (6.25,17.75) -- (6.25,13.5);
\draw [line width=1pt, ->, >=Stealth] (5.75,15) -- (6.75,15);
\draw [line width=1pt, short] (5.75,15) -- (5.75,13.25);
\end{circuitikz}
}%
\end{figure}
\column{.6\textwidth}
理想变压器:$L_1\,,L_2\to\infty$, $k=1$

理想变压器具有阻抗变换作用。变换关系(反射电阻):
$$
Z_L'=n^2Z_L=\left(\frac{N_1}{N_2}\right)^2 Z_L=\frac{L_1}{L_2} Z_L
$$

其中变压比$n=\frac{N_1}{N_2}=\sqrt{\frac{L_1}{L_2}}$
    \end{columns}
\end{frame}


\begin{frame}{非理想阻容感(寄生效应)}
\end{frame}

\begin{frame}{非理想阻容感(寄生效应)}
\end{frame}

\begin{frame}{非理想晶体管(寄生效应)}
\end{frame}


\section{思路和方法}
\begin{frame}{网络参量矩阵}
    \only<1>{虽然是电电1内容,但在电电2考试中依然有用。网络参量矩阵定义:}
    \begin{columns}
        \column{.5\textwidth}
        \begin{center}
            Z参量:阻抗
        \end{center}
    $$
    \begin{bmatrix}
        v_1\\
        \boxed{v_2}\\
    \end{bmatrix}=\begin{bmatrix}
    z_{11} & z_{12} \\
    \boxed{z_{21}} & z_{22}
    \end{bmatrix}\begin{bmatrix}
        \boxed{i_1}\\
        i_2\\
    \end{bmatrix}
    $$
    \begin{center}
            H参量:混合
        \end{center}
    $$
    \begin{bmatrix}
        v_1\\
        \boxed{i_2}\\
    \end{bmatrix}=\begin{bmatrix}
    h_{11} & h_{12} \\
    \boxed{h_{21}} & h_{22}
    \end{bmatrix}\begin{bmatrix}
        \boxed{i_1}\\
        v_2\\
    \end{bmatrix}
    $$
    
        \column{.5\textwidth}
        \begin{center}
            Y参量:导纳
        \end{center}
    $$
    \begin{bmatrix}
        i_1\\
        \boxed{i_2}\\
    \end{bmatrix}=\begin{bmatrix}
    y_{11} & y_{12} \\
    \boxed{y_{21}} & y_{22}
    \end{bmatrix}\begin{bmatrix}
        \boxed{v_1}\\
        v_2\\
    \end{bmatrix}
    $$
    \begin{center}
            G参量:混合
        \end{center}
    $$
    \begin{bmatrix}
        i_1\\
        \boxed{v_2}\\
    \end{bmatrix}=\begin{bmatrix}
    g_{11} & g_{12} \\
    \boxed{g_{21}} & g_{22}
    \end{bmatrix}\begin{bmatrix}
        \boxed{v_1}\\
        i_2\\
    \end{bmatrix}
    $$
\end{columns}
\vspace{0.5cm}
\only<1>{
\textcolor{TsinghuaPurple}{\bf{记忆方法:21元素表示放大器类型,混合的混是三点水,电流放大}}

也可以用hi,gv进行记忆}

\only<2>{输入输出阻抗/导纳计算:
    $$
    w_{in}=p_{11}-\frac{p_{12}p_{21}}{p_{22}+w_L} \quad w_{out}=p_{22}-\frac{p_{12}p_{21}}{p_{11}+w_S}
    $$
}
\only<3>{
    有源性判断:\textcolor{red}{负阻有源性}和\textcolor{blue}{受控源有源性}
    $$
    \text{有源} \Leftrightarrow \textcolor{red}{\Re{p_{11}}<0 \text{或} \Re{p_{22}}<0} \text{或}\textcolor{blue}{ |p_{21}+p_{12}^\ast|^2>4\Re{p_{11}}\Re{p_{22}}}\Leftrightarrow P^T+P^\ast\text{半正定}
    $$
}
\end{frame}
\begin{frame}{网络参量矩阵}
    \begin{columns}
        \column{.7\textwidth}
    ABCD矩阵:一边当作负载,一边当作输出;本征增益就是开路电压/短路电流对应的增益。
    $$
    \begin{bmatrix}
        v_1\\
        i_1\\
    \end{bmatrix}=\begin{bmatrix}
    A & B \\
    C & D
    \end{bmatrix}\begin{bmatrix}
        v_{2}\\
        \mathbf{\textcolor{red}{-}}i_{2}\\
    \end{bmatrix}
    =\begin{bmatrix}
    A & B \\
    C & D
    \end{bmatrix}\begin{bmatrix}
        v_{out}\\
        i_{out}\\
    \end{bmatrix}
    $$
    \begin{columns}
        \column{.5\textwidth}
        本征电压增益$\textcolor{red}{g_{21}}$
        $$A_{v0}=\frac{v_{out}}{v_{in}}\Big|_{i_{out}=0}=\frac{1}{A}$$
        本征跨阻增益$\textcolor{red}{z_{21}}$
        $$R_{m0}=\frac{v_{out}}{i_{in}}\Big|_{i_{out}=0}=\frac{1}{C}$$
        \column{.5\textwidth}
        本征跨导增益$\textcolor{red}{-y_{21}}$
        $$G_{m0}=\frac{i_{out}}{v_{in}}\Big|_{v_{out}=0}=\frac{1}{B}$$
        本征电流增益$\textcolor{red}{-h_{21}}$
        $$A_{i0}=\frac{i_{out}}{i_{in}}\Big|_{v_{out}=0}=\frac{1}{D}$$
        \end{columns}
    \column{.3\textwidth}
    \begin{figure}[!ht]
\centering
\resizebox{1\textwidth}{!}{%
\begin{circuitikz}
\tikzstyle{every node}=[font=\LARGE]
\draw [ line width=1pt ] (27.5,22.5) rectangle  node {\LARGE \textit{LTI}} (31.25,20);
\draw [ line width=1pt](27.5,22) to[short, -o] (26.25,22) ;
\draw [ line width=1pt](27.5,20.5) to[short, -o] (26.25,20.5) ;
\draw [ line width=1pt](31.25,22) to[short, -o] (32.5,22) ;
\draw [ line width=1pt](31.25,20.5) to[short, -o] (32.5,20.5) ;
\draw [->, >=Stealth] (26.25,21.75) -- (26.25,20.75);
\draw [->, >=Stealth] (32.5,21.75) -- (32.5,20.75);
\draw [->, >=Stealth] (26.25,22.5) -- (27.25,22.5);
\draw [->, >=Stealth] (32.5,22.5) -- (31.5,22.5);
\node [font=\LARGE] at (25.75,21.25) {$v_1$};
\node [font=\LARGE] at (33,21.25) {$v_2$};
\node [font=\LARGE] at (26.75,23) {$i_1$};
\node [font=\LARGE] at (31.75,23) {$i_2$};
\draw [line width=3pt, ->, >=Stealth] (29.25,19.25) -- (29.25,17.5);
\draw [ line width=1pt ] (27.5,17) rectangle  node {\LARGE \textit{LTI}} (31.25,14.5);
\draw [ line width=1pt](27.5,16.5) to[short, -o] (26.25,16.5) ;
\draw [ line width=1pt](27.5,15) to[short, -o] (26.25,15) ;
\draw [ line width=1pt](31.25,16.5) to[short, -o] (32.5,16.5) ;
\draw [ line width=1pt](31.25,15) to[short, -o] (32.5,15) ;
\draw [->, >=Stealth] (26.25,16.25) -- (26.25,15.25);
\draw [->, >=Stealth] (32.5,16.25) -- (32.5,15.25);
\draw [->, >=Stealth] (26.25,17) -- (27.25,17);
\node [font=\LARGE] at (25.75,15.75) {$v_{in}$};
\node [font=\LARGE] at (33,15.75) {$v_{out}$};
\node [font=\LARGE] at (26.75,17.5) {$i_{in}$};
\node [font=\LARGE] at (31.75,17.5) {$i_{out}$};
\draw [->, >=Stealth] (31.5,17) -- (32.5,17);
\end{circuitikz}
}%

\end{figure}
$$
\small{
\begin{cases}
    i_{in}=i_1\\
    v_{in}=v_1\\
    i_{out}=-i_2\\
    v_{out}=v_2
\end{cases}}
$$
    \end{columns}
\end{frame}
\begin{frame}{网络参量矩阵}
    网络参量矩阵可以用于快速求解传递函数。

    \only<1>{使用zyhg矩阵:

    
        \begin{columns}
            \column{.45\textwidth}
            (I) 写出单向网络的传递函数
            $$
            \begin{aligned}
            H_{v\text{单向}}&=\frac{1/g_{11}}{R_S+1/g_{11}}\cdot g_{21} \cdot\frac{g_{21}R_L}{R_L+g_{22}}\\
            &=\frac{(1/R_S)g_{21}R_L}{(1/R_S+G_{11})(R_L+g_{22})}
            \end{aligned}
            $$
            (II) 双向化:分母减去$p_{12}p_{21}$
            \column{.05\textwidth}

            \column{0.5\textwidth}
    \centering
        \begin{figure}[!ht]
            \centering
    \resizebox{1\textwidth}{!}{%
% \resizebox{1\textwidth}{!}{%
\begin{circuitikz}
\tikzstyle{every node}=[font=\LARGE]
\draw [ line width=1pt ] (27.5,22.5) rectangle  node {\LARGE \textit{LTI}} (31.25,20);
\draw [ line width=1pt](27.5,22) to[short, -o] (26.25,22) ;
\draw [ line width=1pt](27.5,20.5) to[short, -o] (26.25,20.5) ;
\draw [ line width=1pt](31.25,22) to[short, -o] (32.5,22) ;
\draw [ line width=1pt](31.25,20.5) to[short, -o] (32.5,20.5) ;
\draw [->, >=Stealth] (26.25,21.75) -- (26.25,20.75);
\draw [->, >=Stealth] (32.5,21.75) -- (32.5,20.75);
\draw [->, >=Stealth] (26.25,22.5) -- (27.25,22.5);
\draw [->, >=Stealth] (32.5,22.5) -- (31.5,22.5);
\node [font=\LARGE] at (25.75,21.25) {$v_1$};
\node [font=\LARGE] at (33,21.25) {$v_2$};
\node [font=\LARGE] at (26.75,23) {$i_1$};
\node [font=\LARGE] at (31.75,23) {$i_2$};
\draw [ fill={rgb,255:red,202; green,240; blue,254} , dashed] (25,18.75) rectangle  (33.75,14.25);
\draw (26.25,17.5) to[R,l={ \LARGE $g_{11}$}] (26.25,15);
\draw (28.75,17.5) to[american voltage source,l={ \LARGE $g_{21}v_1$}] (28.75,15);
\draw (26.25,15) to[short, -o] (23.75,15) ;
\draw (26.25,17.5) to[short, -o] (23.75,17.5) ;
\draw (28.75,17.5) to[R,l={ \LARGE $g_{22}$}] (32.5,17.5);
\draw (32.5,17.5) to[short, -o] (35,17.5) ;
\draw (28.75,15) to[short, -o] (35,15) ;
\draw (18.75,17.5) to[american voltage source,l={ \LARGE $v_s$}] (18.75,15);
\draw (18.75,17.5) to[R,l={ \LARGE $R_S$}] (21.25,17.5);
\draw (21.25,17.5) to[short, -o] (22.5,17.5) ;
\draw (18.75,15) to[short, -o] (22.5,15) ;
\draw (22.5,17.5) to[short] (23.75,17.5);
\draw (22.5,15) to[short] (23.75,15);
\draw (35,17.5) to[short] (36.25,17.5);
\draw (35,15) to[short] (36.25,15);
\draw (37.5,17.5) to[R,l={ \LARGE $R_L$}] (37.5,15);
\draw (37.5,17.5) to[short, -o] (36.25,17.5) ;
\draw (37.5,15) to[short, -o] (36.25,15) ;
\draw [ dashed] (17.5,18.75) rectangle  (21.25,14.25);
\draw [ dashed] (36.75,18.75) rectangle  (41.25,14.25);
\draw [->, >=Stealth] (38.75,17.25) .. controls (39.25,16.25) and (39.25,16.5) .. (38.75,15.25) ;
\node [font=\LARGE] at (39.5,16.25) {$v_o$};
\draw [->, >=Stealth] (23.75,17) -- (23.75,15.5);
\node [font=\LARGE] at (23.25,16.25) {$v_1$};
\draw  (17.5,12.5) rectangle  node {\LARGE \textit{input}} (20,11.25);
\draw  (21.75,12.5) rectangle  node {\LARGE \textit{S/in}} (24.5,11.25);
\draw  (28,12.5) rectangle  node {\LARGE $p_{21}$} (30.75,11.25);
\draw  (34.25,12.5) rectangle  node {\LARGE \textit{out/L}} (37,11.25);
\draw  (38.75,12.5) rectangle  node {\LARGE \textit{output}} (41.25,11.25);
\draw  (47.25,12.75) rectangle (47.25,12.75);
\draw [line width=1pt, ->, >=Stealth] (20.25,12) -- (21.5,12);
\draw [line width=1pt, ->, >=Stealth] (25,12) -- (27.5,12);
\draw [line width=1pt, ->, >=Stealth] (31.25,12) -- (33.75,12);
\draw [line width=1pt, ->, >=Stealth] (37.25,12) -- (38.5,12);
\draw [line width=1pt, short] (25,19) -- (27.25,20);
\draw [line width=1pt, short] (31.5,20) -- (33.75,19);
\end{circuitikz}
}%

\end{figure}

\end{columns}
$$
H_{v\text{双向}}=\frac{(1/R_S)g_{21}R_L}{(1/R_S+g_{11})(R_L+g_{22})-g_{12}g_{21}}
$$
    
    }
    \only<2>{对于梯形网络,可以使用ABCD矩阵。
\begin{columns}
    \column{.5\textwidth}
\begin{figure}[!ht]
\centering
\resizebox{.5\textwidth}{!}{%
\begin{circuitikz}
\tikzstyle{every node}=[font=\LARGE]
\draw (25,23.75) to[short, -o] (26.75,23.75) ;
\draw (25,23.75) to[short, -o] (23.25,23.75) ;
\draw (23.75,26.25) to[R,l_={ \LARGE \textit{Z}}] (26.25,26.25);
\draw (23.75,26.25) to[short, -o] (23.25,26.25) ;
\draw (26.25,26.25) to[short, -o] (26.75,26.25) ;
\end{circuitikz}
}%
\end{figure}
$$
\begin{bmatrix}
1 & Z \\
0 & 1
\end{bmatrix}
$$
  \column{.5\textwidth}
    \begin{figure}[!ht]
\centering
\resizebox{.5\textwidth}{!}{%
\begin{circuitikz}
\tikzstyle{every node}=[font=\LARGE]
\draw (25,26.25) to[R,l={ \LARGE \textit{Y}}] (25,23.75);
\draw (25,26.25) to[short, -o] (26.75,26.25) ;
\draw (25,23.75) to[short, -o] (26.75,23.75) ;
\draw (25,26.25) to[short, -o] (23.25,26.25) ;
\draw (25,23.75) to[short, -o] (23.25,23.75) ;
\end{circuitikz}

}%
\end{figure}
$$
\begin{bmatrix}
1 & 0 \\
Y & 1
\end{bmatrix}
$$
\end{columns}
拆成梯形网络后将各个部件的ABCD乘起来即可。然后利用
$$
H_{v}=\frac{1}{A},G_{m}=\frac{1}{B},R_{m}=\frac{1}{C},H_{i}=\frac{1}{D}
$$
得到总的传递函数。
    }

\end{frame}
\end{document}